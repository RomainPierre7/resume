%-------------------------------------------------------------------------------
%	SECTION TITLE
%-------------------------------------------------------------------------------
\cvsection{Projets}


%-------------------------------------------------------------------------------
%	CONTENT
%-------------------------------------------------------------------------------
\begin{cventries}

%---------------------------------------------------------
  \cventry
    {Projet académique de compilation} % Job title
    {ENSEIRB-MATMECA} % Organization
    {Bordeaux (33), FRANCE} % Location
    {Novembre 2023} % Date(s)
    {
      \begin{cvitems} % Description(s) of tasks/responsibilities
        \item{Réalisation d'un compilateur vers du code 3-adresses à l'aide de Lex et Yacc.}
        \item{ravail en équipe de 2 durant 1 mois.}
      \end{cvitems}
    }

%---------------------------------------------------------
  \cventry
    {Projet académique de système de gestion de base de données (S.G.B.D.)} % Job title
    {} % Organization
    {} % Location
    {Novembre 2023} % Date(s)
    {
      \begin{cvitems} % Description(s) of tasks/responsibilities
        \item{Conception d'une base de données pour une application de covoiturage sous PostgreSQL. Affichage web avec PHP.}
        \item{Travail en équipe de 4 durant 1 mois.}
      \end{cvitems}
    }

%---------------------------------------------------------
  \cventry
    {Projet académique de programmation orientée objet (P.O.O.)} % Job title
    {} % Organization
    {} % Location
    {Novembre 2023} % Date(s)
    {
      \begin{cvitems} % Description(s) of tasks/responsibilities
        \item{Sujet pas encore distribué.}
      \end{cvitems}
    }

%---------------------------------------------------------
  \cventry
    {Projet académique de programmation fonctionnelle} % Job title
    {} % Organization
    {} % Location
    {Mars 2023} % Date(s)
    {
      \begin{cvitems} % Description(s) of tasks/responsibilities
        \item{Réalisation d'un Tower Defense en Javascript, utilisant le paradigme de la programmation fonctionnelle.}
        \item{Travail en équipe de 4 durant 2 mois.}
      \end{cvitems}
    }

%---------------------------------------------------------
  \cventry
    {Projet académique de programmation impérative et logiciel} % Job title
    {} % Organization
    {} % Location
    {Mars 2023} % Date(s)
    {
      \begin{cvitems} % Description(s) of tasks/responsibilities
        \item{Réalisation d'un jeu de plateau en C avec implémentation client/serveur.}
        \item{Travail en équipe de 4 durant 2 mois.}
      \end{cvitems}
    }

%---------------------------------------------------------
  \cventry
    {Projet académique d'algorithmique} % Job title
    {} % Organization
    {} % Location
    {Novembre 2022} % Date(s)
    {
      \begin{cvitems} % Description(s) of tasks/responsibilities
        \item{Réalisation d'un jeu en C.}
        \item{Travail en équipe de 2 durant 2 mois.}
      \end{cvitems}
    }

%---------------------------------------------------------
  \cventry
    {Projet académique de traitement de l'information} % Job title
    {} % Organization
    {} % Location
    {Novembre 2022} % Date(s)
    {
      \begin{cvitems} % Description(s) of tasks/responsibilities
        \item{Réalisation d'une analyse en composante principale (A.C.P.) en Python, portant sur les facteurs de sélection en université américaine.}
        \item{Travail en équipe de 4 durant 1 mois.}
      \end{cvitems}
    }

%---------------------------------------------------------
  \cventry
    {Travail d'initiative personnelle encadré (T.I.P.E.)} % Job title
    {C.P.G.E. Lycée Thiers} % Organization
    {Marseille (13), FRANCE} % Location
    {2020-2022} % Date(s)
    {
      \begin{cvitems} % Description(s) of tasks/responsibilities
        \item{Étude de la propagation des feux de forêt et phénomène de percolation (automates cellulaires). Réalisé en Pyhton.}
      \end{cvitems}
    }

%---------------------------------------------------------
  \cventry
    {Projet personnel de développement mobile} % Job title
    {Personnel} % Organization
    {FRANCE} % Location
    {Janvier 2023} % Date(s)
    {
      \begin{cvitems} % Description(s) of tasks/responsibilities
        \item{Développement d'une application Android en Kotlin pour estimer son alcoolémie.}
        \item{Projet réalisé seul sur Android Studio.}
      \end{cvitems}
    }

%---------------------------------------------------------

\end{cventries}