%-------------------------------------------------------------------------------
%	SECTION TITLE
%-------------------------------------------------------------------------------
\cvsection{Projets}


%-------------------------------------------------------------------------------
%	CONTENT
%-------------------------------------------------------------------------------
\begin{cventries}

%---------------------------------------------------------
  \cventry
    {Projets académiques} % Job title
    {ENSEIRB-MATMECA} % Organization
    {Bordeaux (33), FRANCE} % Location
    {2022 - Aujourd'hui} % Date(s)
    {
      \begin{cvitems} % Description(s) of tasks/responsibilities
        \item{Projet de compilation.}
        \item{Projet de système de gestion de base de données (S.G.B.D.).}
        \item{Projet de programmation orientée objet (P.O.O.).}
        \item{Projet de programmation fonctionnelle.}
        \item{Projet de programmation impérative}
        \item{Projet d'algorithmique.}
      \end{cvitems}
    }

%---------------------------------------------------------
  \cventry
    {Projet personnel} % Job title
    {} % Organization
    {} % Location
    {Janvier 2023} % Date(s)
    {
      \begin{cvitems} % Description(s) of tasks/responsibilities
        \item{Projet de développement mobile.}
      \end{cvitems}
    }

%---------------------------------------------------------
  \cventry
    {Travail d'initiative personnelle encadré (T.I.P.E.)} % Job title
    {C.P.G.E. Lycée Thiers} % Organization
    {Marseille (13), FRANCE} % Location
    {2020-2022} % Date(s)
    {
      \begin{cvitems} % Description(s) of tasks/responsibilities
        \item{Étude de la propagation des feux de forêt et phénomène de percolation (automates cellulaires). Réalisé en Pyhton.}
      \end{cvitems}
    }

%---------------------------------------------------------


\end{cventries}
